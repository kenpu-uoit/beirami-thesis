\newcommand{\attr}{\mathrm{attr}}

\section{Problem Definition}

\begin{definition}[Temporal database]
    Let $r_1, r_2, \dots r_n$ be the tables
    in a database $D$.  Denote the attributes
    of each table as $\attr(r_i)$.

    A temporal table, denoted $r_i^T$, is
    a table with the attributes:

    $$\attr(r_i^T) = \{\mathrm{updates}, \mathrm{deleted}\} \cup \attr(r_i)$$

    By a temporal database, we mean the database obtained by augmenting
    $D$ by the temporal tables:

    $$ D^T = D \cup \{r_i^T: r_i\in D\} $$
\end{definition}

There is a lot of value in maintaining $D^T$ because it stores the entire update
history of $D$.  Traditionally, this may be seen as an overly burdening task,
but that is no longer the case with increasingly affordable cloud storage
solutions.  Furthermore, the update history provides many valuable insights into
the dynamics of data. We are motivated to support queries for the database at a
specific time.

\begin{definition}[Snapshots and queries]
    Given a temporal table $r^T$, and a timestamp $t$, we denote
    $r(t)$ to be the table instance obtained by applying the updates
    in $r^T$ where $r^T[\mathrm{updates}] \leq t$.

    A snapshot is a materialized version of $D(t) = \{r_1(t), r_2(t), \dots,
    r_n(t)\}$.  A snapshot query is an arbitrary relational query on $D(t)$.
\end{definition}

We can construct the snapshots using simple windowing functions (as in supported
by PostgreSQL).


\vspace{1em}

{\small
\begin{tabular}{|l|} \hline
    $\mathrm{snapshot}(r, t)$ = \\
    \verb|| \textsc{With} $T$ \textsc{as} ( \\
    \verb| | \textsc{select} id, $\{\mathrm{last\_value}(x) \mathrm{\ as\ } x:
    x\in\attr(r)\}$ \textsc{over} $W$ \\
    \verb| | \textsc{from} $r^T$ \\
    \verb| | \textsc{where} updates $\leq t$ \\
    \verb| | \textsc{window} $W$ \textsc{as} 
             \textsc{partition by} id \textsc{order by} updates\\
    \verb|| ) \\
    \verb|| \textsc{select} id, $\{x: x\in\attr(r)\}$ \textsc{from} $T$ \\
    \verb|| \textsc{where not} $T.$deleted \\ \hline
\end{tabular}
}

\vspace{1em}

The query $\mathrm{snapshot}(r, t)$ computes the snapshot of $r$ at timestamp
$t$ by applying the latest update of each tuple up to timestamp $t$, while
removing tuples that have been deleted.

\begin{prop}
    Assume that the tables are updated at a constant rate over time,
    then the complexity of $\mathrm{snapshot}(r, t)$ is 
    $$\mathcal{O}(|\{x: x\in r^T\mathrm{\ and\ } x.\mathrm{updates} \leq t\}|)
    \simeq \mathcal{O}(t)$$
\end{prop}

\subsection{Query answering using snapshots}

In order to answer a query $Q(t)$ on $D(t)$, we propose to first precompute
snapshots $\mathrm{snapshot}(r, t)$ and then evaluate $Q$.  If some snapshots
are precomputed and materialized, then we can save on the computational cost in
answering the query $Q(t)$.

\begin{prop}
    Suppose we have materialized $\mathrm{snapshot}(r, s)$.  Then
    $\mathrm{snapshot}(r, t)$ can be computed with complexity:

    $$\mathcal{O}(|\{x: x\in r^T\mathrm{\ and\ } x.\mathrm{updates} \in [s,
    t]\}|) \simeq \mathcal{O}(|s-t|)$$
\end{prop}

\subsection{Optimal materialization of snapshots}

We define the central problems of this paper, namely computing the timestamps at
which we can materialize the snapshots to best answer a given query workload.
Let $T_Q = \{q_1, q_2, \dots, q_n\}$ be the timestamps of $n$ queries, each 
querying the database at $D(q_i)$.  We define the cost functions as the total
query answering cost given some precomputed and materialized snapshots.

\begin{definition}[Query answering cost]
    Suppose that a single snapshot at time $s$ is materialized, then
    $$\mathrm{cost}(T_q | s) = \sum_{q\in T_q} |q - s|$$

    If we have multiple snapshots at times $S=\{s_1, s_2, \dots, s_m\}$
    materialized, then

    $$\mathrm{cost}(T_q|S) = \sum_{q\in T_q} \min\{|q-s| : s\in S\}$$
\end{definition}

\begin{definition}[Optimal snapshot placement]
    The {\em single snapshot placement} problem is
    to compute the timestamp $s^*$ such that $\mathrm{cost}(T_q | s)$ is
    minimized.

    The {\em $m$-snapshots placement} problem is
    to compute a set of timestamps $S^* = \{s_1, s_2, \dots s_m\}$
    such that $\mathrm{cost}(T_q | S)$ is minimized.
\end{definition}

In the remainder of the paper, we present the algorithms and experimental
evaluation to demonstrate that the optimal snapshot placement problems can be
efficiently solved, and that their solutions yield a practical solution to
support queries of temporal databases.
