\section{Introduction}

Recently, as part of the Big Data movement,
information systems have been designed to archive every bit of 
data over time \cite{tian2015prototype}. In contrast to traditional data
warehouse databases, modern
databases are aimed to store {\em every} update, insert and deletion as a timeline.
This is known as {\em multi-version databases} \cite{sadoghi2014reducing}.

\begin{example}
    Consider the following table {\tt Products} which stores the product
    information.

    \vspace{1em}

    \begin{center}
        \small
        \begin{tabular}{|c|c|c|c|} \hline
            id & part-name & Manufacture & Price \\ \hline
            0010 & Game Console &  Nintendo & \$399 \\ \hline
            0011 & Keyboard &  Microsoft & \$89 \\ \hline
            $\vdots$ & \vdots & \vdots & \vdots \\ \hline
        \end{tabular}
    \end{center}

    \vspace{1em}

    Suppose that there will be frequent updates to the products such as
    addition of new products, renaming of existing products, and price
    adjustments. The {\tt product} table can only store the eventual state of
    the database {\em after} each updates, but not the updates themselves.  The
    update transactions are quite valuable themselves.

    We would like to augment the database schema to make {\tt product} into a
    {\em temporal} table which we will call $\mathtt{product}^T$. The temporal
    table is to store {\em all} modifications to the {\tt product} table.
    Hence, it has the following schema:

    \vspace{1em}
    \begin{center}
        \small
        \begin{tabular}{|c|c|c|c|c|c|} \hline
            :updates: & :deleted: & id & part-name & Manufacture & Price \\ \hline
            00000001 & false & 0010 & Game Console & Nintendo & \$359 \\ \hline
            00000002 & false & 0020 & Monitor & Samsung & \$159 \\ \hline
            00000003 & false & 0010 & Game Console & Nintendo & \$399 \\ \hline
            00000004 & true & 0011 & - & - & - \\ \hline
            $\vdots$ & $\vdots$ & $\vdots$ & $\vdots$ & $\vdots$ &  $\vdots$ \\ \hline
        \end{tabular}
    \end{center}
    \vspace{1em}

    It shows four specific updates done to the {\tt product} table:

    \begin{itemize}
        \item The {\em game console} is discounted to a new price \$359.
        \item A new product {\em Monitor} is added to the table.
        \item The {\em game console} is no longer on sale, so its price is back
            to \$399.
        \item The keyboard (product ID 0011) is discontinued, so it's deleted
            from the table.
    \end{itemize}

\end{example}

