\section{Design of trusted relational tables}
\label{sec:trusted-db}

Based on the problem definition given in 
Section~\ref{sec:problem-def}, we describe the design of the trusted database
management system.  In our design, we embed blockchains in the relational
tables to support immutable transactions and verification.

\subsection{Users, keys and digital signatures}

\newcommand{\public}{\operatorname{public}}
\newcommand{\private}{\operatorname{private}}
\newcommand{\enc}{\operatorname{enc}}

Let $U$ be a collection of {\em users}.  Each instance in $U$ can be a user in
the conventional sense, a mobile device or an Internet of Things (IoT).
Without the loss of generality, we call all such instances {\em users}.  A
user $u\in U$ is uniquely characterized by a pair of public and private keys:
$\public(u)$, $\private(u)$.  The key pairs are typically generated
using the RSA scheme \cite{cormen2009introduction} using large number
of bits (e.g. 4096) to avoid key collisions
even when there are billions of instances in $U$.

Given arbitrary data $x$, the encoding function is given as
$\enc: (x,\private(u))\mapsto \Hat{x}$ where $\Hat{x}$ is a lossless encoding
of $x$, commonly known as the ciphertext.  The ciphertext can be decoded using
the encoding function again, but with the public key:
$$ \enc(\Hat{x}, \public(u)) = x $$

To prove authorship of some data $x$ by user $u$, the following digital signing
scheme is used:

\begin{enumerate}
    \item Compute the hash of $x$ by a fixed hash function: $h(x)$.
    \item Compute the ciphertext of $h(x)$: $\mathrm{sig}(x) = \enc(h(x),
        \private(u))$.
    \item Publish the data, the signature and the public key:
        $\left<x, \mathrm{sig}(x), \public(u)\right>$.
        The published data does not leak the private key of the user.
\end{enumerate}

To verify the authenticity of the authorship from the published data:

\begin{enumerate}
    \item Decode the signature:
        $y = \enc(\mathrm{sig}(x), \public(u))$.
    \item Verify that the decoded signature is the hash value of $x$ using the
        fixed hash function $h$:
        $ h(x) \overset{?}{=} y $
\end{enumerate}

\subsection{Blockchain in relational tables}

We propose a scheme to augment each relational table in a database $D$ with
additional attributes to store the transaction timestamp, a new table signature,
a previous table signature,
and the user public key.  We also need an extra bit flag to indicate if the
transaction is a deletion.  In practice, the deletion bit flag can be padded
to the end of the transaction signature, but for readability, we assume
yet another boolean attribute added to the table.

\newcommand{\attr}{\mathrm{attr}}
For each table $T$ in the database $D$ with attributes $\attr(T)$, we assume
that at least one attribute, $id$, is the row id\footnote{Popular RDBMS such as MySQL
and Postgresql have built {\tt row\_id} to generate row ids.}.
The table $T$ is then augmented with the additional attributes: $(\mathrm{t}, \mathrm{sig},
\mathrm{sig}', \mathrm{pubkey}, \mathrm{del?})$, each is as described above.
The new table with the augmented attributes is written as $\Hat T$.
Note, we keep track of two signatures: $\mathrm{sig}$ and $\mathrm{sig}'$
corresponding to the signatures of the table {\em after} $T(t)$ and {\em before}
$T(t-1)$
the transaction respectively.  This is necessary to ensure the blockchain
integrity \cite{dhillon2017blockchain,wang2018ablockchain}.

\medskip

{\bf Transaction updates}:  \quad Suppose a user $u\in U$ wants to commit a
transaction $\Delta D$.  For each table $T$ involved in the transaction,
let $\Delta T$ be the rows affected (added, modified, or removed).  For each
tuple $x\in\Delta T$, we insert into $\Hat T$ the augmented tuple
$\left<x, i, s, s', \public(u), \mathrm{del?}(x)\right>$ where
$i$ is the current transaction timestamp, $s, s'$ the signatures as
described below, $\public(u)$ the public key of the transaction author, and
finally $\mathrm{del?}(x)$ is the boolean flag indicating if $x$ is removed by
in the transaction.

The signatures are given as:

\begin{itemize}
    \item $s' = x[\mathrm{sig'}]: x[t] = i-1$. This is unique determined as the
        signature of the previous transaction at timestamp $i-1$.
    \item $s = \enc(\mathrm{hash}(\Delta T + s'), \private(u))$. This is the
        digital signature of the transaction data {\em and} the previous
        signature.
\end{itemize}

\medskip

{\bf Verification}: \quad Each table-level transaction $\Delta T$ is uniquely
identified by its timestamp $t$.  It is also uniquely identified by its
signature $\mathrm{sig}_t$.  The verification of the transaction is given as:

$$\mathrm{verify}(\Delta T) = \enc(\mathrm{sig}_t, \public(u)) 
    \overset{?}= \mathrm{hash}(\Delta T + \mathrm{sig}_{t-1}) $$

This allows us to verify all the tables $\Hat T$ in the temporal database.  Any
tampering will be detected by the verification function.


