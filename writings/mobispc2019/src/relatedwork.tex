\section{Related Work}

There has been studies on the trustworthiness. They range from
establishing trust between nodes in a real-time distributed systems
\cite{khayat2017trust} to secret sharing schemes in cloud databases
\cite{dutta2013privacy} and utilizing logfiles for forensic purposes
\cite{sinha2014continuous}. In this project, the assumption is that the
preventive models are not able to stop the adversaries from manipulating the
data of a relational database system, therefore a tamper-evident log table has
been offered to evaluate whether or not the data has been altered. 

        Database audit logs contain valuable information such as any insertions,
deletions, and modifications of the records performed in a database along with
the timestamp of the performed tasks. The United States Department of Defense in
its “Trusted Computer System Evaluation Criteria” document and under requirement
4 pointed out the importance of auditing the transactions in a computer system
in a secure and efficient manner \cite{USDoD1985}. In this document also
protecting the audit logs from modification or destruction has been stressed
out.

        Analyzing audit logs for forensic purposes has been the topic of
research by many scientists, however, since the trustworthiness of the results
from log table analysis has a direct relationship with the authenticity of the
records of the log file, a lot of studies have been carried out to make log
tables tamper-proof. Haber {\it et al.}\cite{haber1991how} proposed a basic
methodology by utilizing timestamping and hash chains in order to make
unmodifiable historical records for digital documents. Peha in
\cite{peha1999electronic} offered a method to detect log tampering by one way
hashing and employing multiple trusted notaries to keep track of all
transactions. The author argued that if any notary decided to falsify the
transaction, the attempt is discovered by other notaries. Snodgrass {\it et al.}
\cite{snodgrass2004Tamper} also offered one-way hashing mechanism and employed
trusted notaries, however in order to enhance the security of the method offered
by Peha, they offered to hash the records with a timestamp of previous
transaction modification. Schneier {\it et al.}\cite{schneier1998cryptoraphic}
offered a cryptographic-based mechanism to create hash chains and make the log
files nearly impossible for the attacker to alter. The validity of the
transactions was also done by trusted third-parties who have the cryptographic
keys.

        However the aforementioned researches might have promising results to
protect the historical records from being compromised by an outsider, but they
have one thing in common which is the role of an insider to carry out malicious
attacks is forgotten. Also hiring third-party software/hardware may bring up a
lot of privacy concerns. Therefore, unlike the mentioned works, not only our
proposed system does not require a third-party notary to attest the authenticity
but also it does not put trust on any user of the system.

        The role of privileged users in acting maliciously in a database has
been discussed by many researchers \cite{crosby2009tamper-evident}
\cite{wagner2018detect}. Liu {\it et al.} offered a network-based auditing
mechanism which also keeps track of privileged users' activity and performs
audit analysis through event correlation. Wagner {\it et al.}
\cite{wanger2017carving} proposed a mechanism to detect database file tampering
by looking for inconsistencies in the database's storage. The authors argued
that the databases follow patterns in storage which even the privileged users
have no access to. Therefore, if a record is deleted maliciously in the log
file, the inconsistency in the storage is evident for a period of time. Unlike
mentioned proposed methods, our system uses inherent to RDBMS tools and does not
require a network-level-auditing mechanism or having access to the server's
storages, therefore it could be easily implemented on remote servers and
relational databases on cloud storages. Also, our proposed system not only
discovers maliciously deletion of the records but also identifies any malicious
modification without any time constraints.

        For the sake of gathering verifiable pieces of evidence, in our proposed
system, any changes to the database regardless of the users' access privileges
need to be tracked. This action could be done by utilizing inherent to DBMS
tools. Fabbri {\it et al.} \cite{fabbri2013select} extensively talked about
SELECT triggers which are inherent RDBMS functions, required database
specifications and efficient implementation techniques for data auditing. Hauger
{\it et al.}\cite{hauger2014information} also discussed the use of triggers in
the database for forensic purposes. Triggers are supported by RDBMS which makes
them a good candidate to be used in our proposed system however, it is naive to
assume that a simple trigger solely is secure enough to be used for forensic
purposes. Therefore we propose to use the Blockchain technology to make
immutable chains of transactions that are captured by database triggers.

        The first attempt to use chained hashes for securing data from tampering
is known to be done by Haber {\it et al.} \cite{haber1991how} where the authors
proposed a methodology to securely timestamp the digital files and create
digital signatures and hash chains. This work was then improved by Schneier {\it
et al.} \cite{schneier1998cryptoraphic} \cite{schneier1999minimizing}
\cite{schneier1999secure} by offering to exchange secret codes with a trusted
third party who is able to verify the authenticity of the chain. They offered a
method to change the shared secret as the new transaction occurs, therefore
since the attackers do not have the previous secret codes, it is impossible for
them to alter any records which were previously added. Our proposed method is
similar to the mentioned works in this aspect that our historical records are
chained together using the digital signatures, however, we use asymmetric
encryption to generate digital signatures that removes the need of the
third-parties to attest the authenticity of the transactions. The asymmetric
encryption method also enables anybody to verify signatures without revealing
the users' secrets.
