\section{Introduction}

In this paper, we study the problem of trusted relational database systems to
support immutable and auditable transactions and efficient processing of large
number of analytical queries.

\subsection{Motivation}

Consider a relational database
used to store purchase orders of various customers and suppliers.  Over the time,
customers and suppliers may update the database: inserting new orders, 
make modifications to previous orders, or deleting existing orders.  Trust is an
exceedingly important aspect of electronic data management. The system must
maintain the provenance of all database transactions in scenarios of disputes.
For example, a customer $C_1$ may claim that an order has been cancelled {\em
before} an agreed deadline, but the
supplier $S_2$ may disagree with {\em when} the order has been cancelled.  One
customer may claim that an order was properly inserted into the database, but
somehow the disputed order is missing.  When such disagreement over the history
of the database arises, we would like to have an audit process for which the
finding is {\em indisputable} and hence {\em trusted} by all parties.

We would like to design a trusted relational database system built on existing
and mature technology (such as modern standard RDBMS\footnote{We use Postgresql
for our implementation.}) to store the transactions such that the entire history
of the database is tamper proof.  Namely, while we may not be able to prevent
unwanted modifications to the database (e.g. unwanted root level access to the
database, or hardware failure), if any modification to the data history will be
revealed by an audit process.  In other words, we can verify {\em trust} the
state of the database if the database is intact and free of tamper.

Equally important to trust, the database must also support efficient evaluation
of analytical queries.  We envision that many users may submit a large number of
queries, which we will refer to as the query workload, to the database.  Each
query may wish to examine the state of the database at a query specific timestamp
in history.  The queries may be simple select queries,
or may have complex data transformations and aggregations.  So another
functional requirement of the trusted database is means of supporting large
query workloads.

\subsection{Problem Definition}
\label{sec:problem-def}

\newcommand{\D}{D}
\newcommand{\U}{U}

Let $\D$ be a relational database.  We are interested at the evolution
of the database, hence the database is parameterized by {\em version} $\D(t)$
where $t$ is a timestamp.  For simplicity, we will assume $t\in\mathbb{N}^+$.
Let $\U$ be the users.  A transaction is a change to the database
(addition, deletion and modification of tuples in $\D$), loosely denoted by
$\Delta\D$.  We also denote the updated database after the $t$-th transaction
submitted by user $u\in\U$ as:
$\D(t+1) = \D(t)+(\Delta\D_t, u)$.

Assuming that the database starts with an empty initial state of $\emptyset$ at
$t=0$, we define the timeline of the database up to some time $t$ as the series
$$ \mathrm{TL}(t) = 
    \left[\begin{array}{c} \emptyset \\ \mathrm{null} \\ 0 \end{array}\right],
    \left[\begin{array}{c} \D_1 \\ u_1 \\ 1 \end{array}\right],
    \left[\begin{array}{c} \D_2 \\ u_2 \\ 2 \end{array}\right]
    \,\dots\,
    \left[\begin{array}{c} \D_{t-1} \\ u_{t-1} \\ t-1 \end{array}\right].
    $$
Namely, $\D_i$ is the $i$-th transaction submitted by user $u_i$ at time $t=i$.

We wish to design a system $\mathbf{DB}$ to manage the timeline $\mathrm{TL}$ with
the following properties:

\begin{itemize}
    \item {\bf Evolving timeline}: for any time $t > 0$, the timeline
        $\mathrm{TL}(t)$ can be completely retrieved from $\mathbf{DB}$.
    \item {\bf Snapshot retrieval}: for any time $t > 0$, the state of the
        database $D$ can be reconstructed from $\mathbf{DB}$.
    \item {\bf Verification}: There exists a function
        $\mathrm{verify}:\mathbf{DB}\mapsto\{\mathrm{true},\mathrm{false}\}$
        any {\em tampered} version of the system $\mathbf{DB}'\not=\mathbf{DB}$
        will be fail the verification.  We make no assumption of access control,
        so the source of tampering may include root level access to
        $\mathbf{DB}$.  Note, we limit our problem to {\em detection}, not
        prevention of tampering.
    \item {\bf Query Workload}: let $q$ be some query defined on a snapshot
        $D(t_q)$.  We call $q_t$ the query timestamp.  The system $\mathbf{DB}$
        can efficiently evaluate a large number of such queries at different
        query timestamps.  The collection of queries is called the {\em query
        workload}, written $Q = \{q_1, q_2, \dots, q_N\}$.
\end{itemize}
