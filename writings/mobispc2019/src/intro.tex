\section{Introduction}

In this paper, we study the problem of implementing a trusted relational
database system to support immutable and auditable transactions, and efficient
processing of large numbers of analytical queries.

\subsection{Motivation}

Consider a relational database
used to store purchase orders of various customers and suppliers.  Over time,
customers and suppliers may update the database: insert new orders, 
make modifications to previous ones, or delete existing ones.  Trust is an
exceedingly important aspect of electronic data management. The system must
maintain the provenance of all database transactions in scenarios of dispute.
For example, a customer $C_1$ may claim that an order has been cancelled {\em
before} an agreed deadline, but the
supplier $S_2$ may dispute the time of cancellation.  There are many such
scenarios in practice.
When such disagreements over the history
of the database arise, we would like to have an audit process for which the
finding is {\em indisputable} and hence {\em trusted} by all parties.

We would like to design a trusted relational database system built on existing
and mature technology (such as modern standard RDBMS\footnote{We use Postgresql
for our implementation.}) to store the transactions in such a way that the entire history
of the database is tamper proof.  Namely, while we may not be able to prevent
unwanted modifications to the database (e.g., malicious root users 
or accidental hardware failures), {\em any} modification to the data and its history will be
revealed by an audit process.  In other words, we can {\em trust} the
state of the database by verifying that the database is intact and free of tamper.

Equally important to trust, the database must also support efficient evaluation
of analytical queries.  We envision that many users may submit a large number of
queries, which we will refer to as the {\em query workload}, to the database.  Each
query examines the state of the database at a query specific timestamp.  
Each query may be a simple select query, or one that contains complex data
transformations and aggregations.  So another
functional requirement of the trusted database is to support 
query workloads on a large scale.

\subsection{Problem Definition}
\label{sec:problem-def}

\newcommand{\D}{D}
\newcommand{\U}{U}

Let $\D$ be a relational database.  We are interested in the evolution
of the database, hence the database is parameterized by {\em versions}. 
A database state at some version is given by $\D(t)$
where $t$ is a timestamp.  For simplicity, we will assume $t\in\mathbb{N}^+$.
Let $\U$ be the users.  A transaction is a change applied to the database
(addition, deletion and modification of tuples in $\D$), loosely denoted by
$\Delta\D$.  We also denote the updated database after transaction $\Delta D_t$
submitted by user $u\in\U$ as:
$\D(t+1) = \D(t)+(\Delta\D_t, u)$

When the user is understood, we may write
$\D(t+1) = \D(t) + \Delta D_t$

Assuming that the database starts with an empty initial state, $\emptyset$, at
$t=0$, we define the timeline of the database up to some time $t$ as the series
$\mathrm{TL}(t)$ given by:

$ \mathrm{TL}(t) = 
    \left[\begin{array}{c} \emptyset \\ \mathrm{null} \\ 0 \end{array}\right],
    \left[\begin{array}{c} \D_1 \\ u_1 \\ 1 \end{array}\right],
    \left[\begin{array}{c} \D_2 \\ u_2 \\ 2 \end{array}\right]
    \,\dots\,
    \left[\begin{array}{c} \D_{t-1} \\ u_{t-1} \\ t-1 \end{array}\right].
    $

Namely, $\D_i$ is the $i$-th transaction submitted by user $u_i$ at time $t=i$.

We wish to design a system $\mathbf{DB}$ to manage the timeline $\mathrm{TL}$ with
the following properties:

\begin{itemize}
    \item {\bf Timeline retrieval}: for any time $t > 0$, the timeline
        $\mathrm{TL}(t)$ can be completely retrieved from $\mathbf{DB}$.
    \item {\bf Snapshot retrieval}: for any time $t > 0$, the state of the
        database $D$ can be reconstructed from $\mathbf{DB}$.
    \item {\bf Verification}: There exists a function
        $\mathrm{verify}:\mathbf{DB}\mapsto\{\mathrm{true},\mathrm{false}\}$
        that performs the verification of the system such that
        any {\em tampered} version of the system $\mathbf{DB}'\not=\mathbf{DB}$
        will be fail the verification.  We make no assumption regarding access
        control to the storage,
        so the source of tampering may include root level access to
        $\mathbf{DB}$.  Thus, we limit our problem to {\em detection}, not
        prevention of tampering.
    \item {\bf Query Workload}: let $q$ be some query defined on a snapshot
        $D(t_q)$.  We call $t_q$ the query timestamp.  The system $\mathbf{DB}$
        can efficiently evaluate a large number of such queries with different
        query timestamps.  The collection of queries is called the {\em query
        workload}, written $Q = \{q_1, q_2, \dots, q_N\}$.
\end{itemize}
