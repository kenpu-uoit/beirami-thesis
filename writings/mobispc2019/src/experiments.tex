\begin{figure}[t]
\centering
\subfloat[Exact optimal cost v.s. heuristic approximation using clustering]{
    \includegraphics[width=0.45\linewidth]{figs/dynamic_vs_clustering.jpg}
    \label{fig:place-1}
}
\hfill
\subfloat[Approximation quality of 300 iterations vs 5 iterations] {
    \includegraphics[width=0.45\linewidth]{figs/compare_clustering_iterations.png}
    \label{fig:place-2}
}
    \caption{Quality of the approximate snapshot placements using clustering}
    \label{fig:place}
\end{figure}

\section{Evaluation and experiments}

We have conducted a number of experiments to evaluate the effectiveness of the
proposed algorithms.

A synthetic query workload is generated against a database with over 1
million tuples.  Figure~\ref{fig:query-answer} shows the benefit of having
optimally placed snapshots.  Figure~\ref{fig:query-answer-1} shows the effect of
having just a single snapshot.  One can see that the cost is minimal when the
snapshot is at the median of the query workload of 200.
Figure~\ref{fig:query-answer-2} shows the benefit of having more snapshots
optimally placed.  We see that 40 snapshots, the cost is reduced to less than
1/50 compared to just one snapshot.  Figure~\ref{fig:query-answer-3} shows the
relative performance of different snapshot placement strategies.  We see that
random placement and fixed interval placements are significant worse than the
optimal placement strategy which reduced the cost by over 15 times.

We have compared the runtime performance of the optimal snapshot placement using
dynamic programming and clustering based heuristics.  Figure~\ref{fig:runtime-1}
compares the runtime performance of dynamic programming and clustering with 300
iterations.  One can see that the clustering heuristics scales extremely well
with increasing query workload size.  To further speedup the snapshot placement,
we can reduce to fewer iterations as shown in Figure~\ref{fig:runtime-2}.

The heuristic approach is highly effective in obtaining near optimal snapshot
placements. Figure~\ref{fig:place-1} compares the costs of optimal placement and
approximate placements using 300 iterations.  The increase in the cost is less
than 2%.  Even with just 5 iterations, as shown in Figure~\ref{fig:place-2}, the
resulting approximate is very close to the exact optimal placement.
