\section{Conclusion}

We have presented the problem and solutions to build a data management system
that supports:

\begin{itemize}
\item Transaction data processing by large number of user and devices.
\item The trust is maintained using embedded blockchains associated with
relational tables.
\item The blockchain provides verifiability of trust in the data.
\end{itemize}

We are further motivated to support query workloads consisting of analytical
queries at different query timestamps.  To achieve high efficiency, we propose
to select optimal snapshots at various timestamps in such a way that the overall
query evaluation cost is minimized.  We have formulated the problem as a well
defined optimization problem, and presented its exact solution using dynamic
programming.  To reduce the optimization overhead, we constructed a clustering
based heuristic algorithm.  The experimental evaluation shows that materialize
snapshots greatly improves query throughput (by a factor of 50 times), and the
heuristic approximation of snapshot placement is very close (within 2%) to the
exact optimal placements.

\medskip

{\em Future work.}\ There are a number of directions we plan to further this
research program.

\begin{itemize}
\item Our approach is currently limited to the relational model.  We plan to
generalize it to other data models including semi-structured and graph data
models.
\item We are able to obtain the (near) optimal snapshot placements for static
query workload.  We plan to explore adaptive snapshots placements to handle
dynamic query workload in an online fashion.
\item Blockchain is inherently distributed.  We envision that the blockchain's
distributed consensus protocol can be adopted to manage trust in a distributed
fashion, and possibly make it possible to perform database repair on tampered
database instances.
\end{itemize}
