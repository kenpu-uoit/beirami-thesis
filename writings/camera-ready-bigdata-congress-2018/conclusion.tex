\section{Conclusion and Future Work}

In this paper, we have presented some results obtained toward optimally
supporting temporal relational queries of databases that store the timelines of
its relational tables.  In order to avoid recomputation of the database states
while making use of the storage space efficiently, our solution is to
materialize $m$ snapshots at well-chosen timestamps.

We have constructed a model to describe the query answering cost, and this cost
model allows us to formulate the $m$-snapshot placement problem as an
optimization problem.  We showed that dynamic programming can be used to solve
the problem {\em exactly}.  Our experimental evaluation demonstrates that our
cost model agrees with relational database systems, and that our snapshot
placements improve query processing significantly.

This is on-going research. As future work, we will be investigating
approximation methods to further speed up the snapshot placement calculuation.
We also wish to investigate maintaining snapshot placement dynamically to cope
with a dynamic query workload.
