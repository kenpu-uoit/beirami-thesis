\chapter{Introduction}
\label{chap:introduction}

\section{Motivation}
Today, data is seen as the lifeblood of organizations that is helping them to make strategic decisions more efficient and perform the operations faster. Organizations store their highly confidential data, such as financial documents, customer information, medical records and more in the form of data records in a database and later, use them for their sensitive operations. However, in our current connected era, cyber-security is one of the biggest challenges of organizations. Databases, could be accessible on the public networks where adversaries utilize hacking techniques to manipulate the data that are stored in them. Offline databases also are not safe as the attacks might be carried out within the organization by privileged users of the system. The fake data that are the result of such malicious activities, if not identified on the database, may result in irreversible consequences. As a result, organizations and businesses spend a considerable amount of money to utilize cyber-security techniques in order to make their stored data safe and reliable. 

Traditionally, malicious activities on a system including fake data manipulation to the database is prevented by restricting the activities of the users on the system. Also cryptographic techniques such as data encryption or electronically signing the data has proven effective in identifying fake data manipulation. This is primarily done by assigning a pair of cryptographic keys to the users, by which they can securely encrypt or decrypt the data and submit it to the database. Fabricating cryptographic keys is computationally infeasible, hence it is extremely difficult for an outsiders to manipulate data inside the databases without notice. The downside of these techniques is that it requires to fully trust the activities of authenticated users which in turn, brings up a lot of security concerns. Also with ever-increasing complexity of cyber-criminal techniques, each day a new approach to penetrate the database systems is identified. Hence, it is naive to assume that access restriction or cryptographic techniques alone could solve the issue. 

Therefore a system is needed to confirm the reliability of data based on verifiable evidences and not by relying on trust. This requires that the transactions on the database system to be transparent, meaning that for a record in a database throughout its life-cycle, it should be evident that its data has always been generated and modified by official sources. To achieve this, the system not only should be able to identify and report the data generated from unrecognized sources but also it should be able to show the proof of work done by privileged users. By providing proof of work, all users who interfered with data manipulation in a database system are identified and their activity information is reported. 

In this thesis we have developed a system which ensures transparency of activities in a relational database system. Our developed system identifies and reports any malicious data manipulation by outsiders and provides proof of work by referring to the history of transactions for any records stored in the database. History of records are temper-proof and is protected by cryptographic techniques. We also developed a mechanism to handle large query workloads on the historical records. There were four main challenges that needed to be addressed while developing the system: Auditing the transactions on the database system, handling large query workloads on the audit tables, verifying the validness of query results and making the audit tables to be temper-proof. 
\section{Related Work}
\section{Problem Definition}
%--------------------------------------------------------------------
Given a relational database $D$, let $r$ to be the relational table in $D$. Denote attributes of $r$ as $attr(r)$. Assume $attr(r)= [id,m,u, sig(m|u)]$ where $id$ is the id of records in the database, $m = [col_1,...,col_n]$ is $n$ number of data columns in $r$, $u$ is the information of the user who submitted the record and $sig(m|u)$ is the digital signature of the record submitted by $u$. Also let $D^T$ be the temporal database in which the history of the records in $D$ is stored. we denote $r^T$ as the tables in $D^T$. Assume that the attributes of $r^T$ are $attr(r^T) = [id,m,u,sig(m|u),t,d]$ where t and d are the timestamps in which the record was created/updated or deleted respectively.\\

A submitted transaction is said to violate transparency, hence untrustworthy in any of the following scenarios:\\
\textbf{Scenario 1.} Let q be the result of the query $q=\sigma_{(id)}(r)$, which is the record submitted by the user $u$. The result of query is untrustworthy if $\{q[sig(m|u)]: sig(m|u) \in r\} \neq sig(\{q[m]: m \in r \} |u)$. That is, by digitally signing the record $m$ with the $u$'s cryptographic keys, we get a different result than the submitted signature to the table. The reason that this scenario may happen is that: 
\begin{itemize}
	\item  The record was altered accidentally or maliciously.
	\item  A user maliciously claims to be one of the privileged users of the system with fake credentials.
\end{itemize} 
\textbf{Scenario 2.} Let q be the result of query $q=\sigma_{(id)}(r)$. The result of query is untrustworthy if $q[u \vee sig(m|u)] = \emptyset$. In other words, the resulted record was submitted by an anonymous user to the database.\\
\textbf{Scenario 3.} Given a particular timestamp $t_0$, the result of query on the temporal database  $q^T=\sigma_{(id,t=t_0)}(r^T)$ is untrustworthy if $\{q^T[sig(m|u)]: sig(m|u) \in r^T\} \neq \{sig(q^T[m]:m \in r^T\}|u)$ and if $q^T[u \vee sig(m|u)] = \emptyset$. This means that a former transaction on the record that occurred in $t_0$, was submitted illegally. \\
\textbf{Scenario 4.} Given the current timestamp $t_{max}$, let $q^T = \sigma _{(id,max(m|t_{max}): m \in r^T)}(r^T)$ to be the latest version of a record in $D^T$ and $q=\sigma_{id}(r)$ to be the same record in $D$. A record is said to be untrustworthy if $q^T \neq q$ meaning that the latest version of a record in the temporal table does not match the record in a normal table. This include the following cases:
\begin{itemize}
	\item $q^T[m] \neq q[m]$
	\item $q^T[sig(m|u)] \neq q[sig(m|u)]$
	\item $q^T[d|id] \neq \emptyset$ but $q[id] \in r$
	\item $q^T[d|id] = \emptyset$ but $q[id] \notin r$
\end{itemize}


All in all, we argue that the data in a database is said to be transparent if:
\begin{enumerate}
	\item The content of the records match the submitters' digital signature.
	\item No anonymous transaction was submitted to the system.
	\item History of applied transactions is provided for all records. 
	\item Items 1 and 2 are valid for all former transactions on the records.
	\item The latest version of the records in the temporal audit table match the records in the normal table.
\end{enumerate}
%--------------------------------------------------------------------
\section{Contributions}
\section{Outline of the Thesis}
