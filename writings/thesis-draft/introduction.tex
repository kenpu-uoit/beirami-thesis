\chapter{Motivation and Problem Definition}
\label{chap:introduction}

\section{Motivation}
Today, data is seen as the lifeblood of organizations that is helping them to make strategic decisions more efficient and perform the operations faster. Organizations store their highly confidential data, such as financial documents, customer information, medical records and more in the form of data records in a database and later, use them for their sensitive operations. However, in our current connected era, cyber-security is one of the biggest challenges of organizations. Databases, could be accessible on the public networks where adversaries utilize hacking techniques to manipulate the data that are stored in them. Offline databases also are not safe as the attacks might be carried out within the organization by privileged users of the system. The fake data that are the result of such malicious activities, if not identified on the database, may result in irreparable consequences. As a result, organizations and businesses allocate a considerable amount of fund to utilize cyber-security techniques in order to make their stored data safe and reliable. 

Traditionally, malicious activities on database systems including fake data manipulation is prevented by restricting the activities of the users on the system. Also cryptographic techniques such as data encryption or electronically signing the data has proven effective in identifying fake data manipulation. This is primarily done by assigning a pair of cryptographic keys to the users, by which they can securely encrypt or decrypt the data and submit it to the database. Fabricating cryptographic keys is computationally infeasible, hence it is extremely difficult for an outsider to manipulate data inside the databases without notice. The downside of these techniques is that it requires to fully trust the activities of authenticated users which in turn, brings up a lot of security concerns. Also with ever-increasing complexity of cyber-criminal techniques, each day a new approach to penetrate the database systems is identified. Hence, it is naive to assume that access restriction or cryptographic techniques alone could solve the issue. 

Therefore a system is needed to confirm the reliability of data based on verifiable evidences and not by relying on trust. This requires that the transactions on the database system to be transparent for the proposed system, meaning that for a record in a database throughout its life-cycle, it should be verifiable that its data has always been generated and modified by official sources. To achieve this, the system not only should be able to identify and report the data generated from unrecognized sources but also it should be able to show the proof of work done by privileged users. By providing proof of work, all users who interfered with data manipulation in a database system are identified and their activity information is reported. 

In this thesis a system is developed to ensures transparency of activities in a relational database system. The developed system identifies and reports any malicious data manipulation by outsiders and provides proof of work for each record by referring to the history of transactions. History of transactions are temper-proof and is protected by cryptographic techniques. Also a practical solution is provided to enhance the query performance on the historical records that are stored in the tamper-proof temporal audit tables. There were four main challenges that needed to be addressed while developing the system: Auditing the transactions on the database system, handling large query workloads on the audit tables, verifying the validness of query results and making the audit tables to be temper-proof. 
\section{Problem Definition}
%--------------------------------------------------------------------
\subsection{Transaction Transparency}
Given a relational database $D$, let $r$ to be the relational table in $D$. Denote attributes of $r$ as $attr(r)$. Assume $attr(r)= [id,m,u, sig(m|u)]$, where $id$ is the id of records in the database, $m = [col_1,...,col_n]$ is $n$ number of data columns in $r$, $u$ is the information of the user who submitted the record and $sig(m|u)$ is the digital signature of the record submitted by $u$. Also let $D^T$ be the temporal database in which the history of the records in $D$ is stored. we denote $r^T$ as the tables in $D^T$. Assume that the attributes of $r^T$ are $attr(r^T) = [id,m,u,sig(m|u),t,d]$ where $t$ and $d$ are the timestamps in which the record was created/updated or deleted respectively.\\

A submitted transaction is said to violate transparency, hence untrustworthy in any of the following scenarios:\\
\textbf{\emph{Scenario 1.}} Let q be the result of the query $q=\sigma_{(id)}(r)$, which is the record submitted by the user $u$. The result of query is untrustworthy if $\{q[sig(m|u)]: sig(m|u) \in r\} \neq sig(\{q[m]: m \in r \} |u)$. That is, by digitally signing the record $m$ with the $u$'s cryptographic keys, a different result is returned than the submitted signature to the table. The reason that this scenario may happen is that: 
\begin{itemize}
	\item  The record was altered accidentally or maliciously.
	\item  A user maliciously claims to be one of the privileged users of the system with fake credentials.
\end{itemize} 
\textbf{\emph{Scenario 2.}} Let q be the result of query $q=\sigma_{(id)}(r)$. The result of query is untrustworthy if $q[u \vee sig(m|u)] = \emptyset$. In other words, the resulted record was submitted by an anonymous user to the database.\\
\textbf{\emph{Scenario 3.}} Given a particular timestamp $t_0$, the result of query on the temporal database  $q^T=\sigma_{(id,t=t_0)}(r^T)$ is untrustworthy if $\{q^T[sig(m|u)]: sig(m|u) \in r^T\} \neq \{sig(q^T[m]:m \in r^T\}|u)$ and if $q^T[u \vee sig(m|u)] = \emptyset$. This means that a former transaction on the record that occurred in $t_0$, was submitted illegally. \\
\textbf{\emph{Scenario 4.}} Given the current timestamp $t_{max}$, let $q^T = \sigma _{(id,max(m|t_{max}): m \in r^T)}(r^T)$ to be the latest version of a record in $D^T$ and $q=\sigma_{id}(r)$ to be the same record in $D$. A record is said to be untrustworthy if $q^T \neq q$ meaning that the latest version of a record in the temporal table does not match the record in a normal table. This include the following cases:
\begin{itemize}
	\item $q^T[m] \neq q[m]$
	\item $q^T[sig(m|u)] \neq q[sig(m|u)]$
	\item $q^T[d|id] \neq \emptyset$ but $q[id] \in r$
	\item $q^T[d|id] = \emptyset$ but $q[id] \notin r$
\end{itemize}


All in all, we argue that the data in a database is said to be transparent if:
\begin{enumerate}
	\item The content of the records match the submitters' digital signature.
	\item No anonymous transaction was submitted to the system.
	\item History of applied transactions is provided for all records. 
	\item Items 1 and 2 are valid for all former transactions on the records.
	\item The latest version of the records in the temporal audit table match the records in the normal table.
\end{enumerate}
\subsection{Queries on Append-only Temporal Database}
Let $D^T$ be the append-only temporal log table in a relational database system. The data volume in such tables grow rapidly as the only action allowed is to append data in the tables. As a result, querying and snapshot computation in such tables become expensive and inefficient. Thus, a solution is needed to handle query workloads as efficient as possible.
\begin{info} \color{red} (MAYBE I SHOULD WRITE MORE????) \end{info}
%--------------------------------------------------------------------
\section{Thesis Statement and Contributions}
In this dissertation, a practical mechanism is presented to provide data transparency in a relational database system. By utilizing this mechanism, an organization could discover malicious data manipulation on their relational database system. Also, since the proof of work is available for all the queries, if a data is suspected to be malicious but submitted with authentic credentials, the responsible users are identified and reported.

The contributions of this dissertation are:
\begin{itemize}
	\item A practical solution to audit the tables of a database and record transactions were offered. For this purpose, the trigger functionality of relational database systems coupled with cryptographic techniques were used. The system can easily verify the validness of sources by matching cryptographic signatures of submitted records. Having recorded transactions also makes it easier to testify that a record has been manipulated by a valid source throughout its life-cycle in a database. 
	
	\item A mechanism was created to enhance query performance in append-only temporal databases. Archiving the transactions on the database records over time in temporal databases is a common practice in information systems, because it provides the form of a table or a record at a specific time as well as it gives insight into the dynamics of data. Over time, the volume of data increase in these databases which results in decay in the query performance. Offered in this dissertation, multiple number of pre-computed snapshots could be generated and placed at the optimal timestamps of the temporal table's timeline, where subsequent queries could utilize them for materialization.
	
	\item A Blockchain approach was developed to make the archived data tamper-proof. By using blockchain, the proof of work for each record is provided.
\end{itemize}
\section{Outline of the Thesis}
The following chapters of the thesis will cover:
\begin{itemize}
	\item In chapter 2, the related work and the tools and concepts that the proposed system is based on, is covered.
	\item The Approach to utilize the tools outlined in chapter 2 to develop the proposed system is discussed extensively in chapter 3.
	\item In chapter 4, the proposed solutions to enhance query performance in Append-only temporal database is put in experiment and the results are compared. Furthermore, different case scenarios are defined and run on the developed system to evaluate the performance.
	\item Discussions, conclusion, limitations and future work are discussed in chapter 5.
\end{itemize}












