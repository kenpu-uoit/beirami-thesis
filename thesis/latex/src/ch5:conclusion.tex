\chapter{Conclusion}

	\section{Summary}
		In this work, implementation of a Blockchain based temporal database was proposed that works based on Relational Database Management System (RDBMS). The trustworthiness of the records in the temporal relations is verifiable using immutable digital signatures. The temporal relation contains all the updates of a particular relation, therefore it could be used as the source of data provenance for that relation. To implement the system, two major problems were identified: 
		
		\begin{itemize}
			\item Make the historical records stored in the temporal table immutable and verifiable.
			\item Lower the cost of snapshot generation on the temporal database.
		\end{itemize}

		Using Blockchain to add immutability and verifiability for the records of temporal relations, require these relations to have a few security information attributes: the transaction submitters information, the digital signature of the record signed by transaction submitter's cryptoraphic keys and the previous record's digital signature. Creating the digital signature of the records and storing them with the record provides verifiability for that records. Also storing the digital signature of the previous record in current record, creates the chain of records also known as Blockchain of records. The Blockchain which is created by doing this, ensures that the malicious attempts on the temporal database, result in an evident inconsistent data.
		The validation of the chain of records is done by
		\begin{itemize}
			\item validate the authenticity of each record by verifying their digital signatures.

			\item making sure that the current record's \it{'previous signature'} field, matches the previous record's \it{'current signature'} field for all the records in the table.
			
		\end{itemize}
		The process of verifying the chain is said to be linear because there is a need to visit every single record of the temporal relation and verify their trustworthiness.
		
		Having a temporal database as the source of data provenance, enables users to create snapshot of the relation in an specific timestamp. In fact for the purpose of decision making and data analytics such queries are common. In this thesis, the problem of linear time and storage to perform such queries on the temporal databases shown both theoretically and experimentally. In the presence of multiple and concurrent queries on the temporal table, performing snapshot creation queries are inefficient and infeasable. 

		In oder to reduce the cost of snapshot creation queries, the materialization of multiple precomputed snapshots proposed. The proposal advices that keeping the timestamp of previous queries which was performed on the temporal table gives insight into the pattern of queries on the table and identifies hotspots. These hotspots are useful because there is a high chance that the subsequent queries to be performed in those hotspots. Optimal segmentation of the performed queries, creates optimal clusters of the queries and specifies a precomputed snapshot for each cluster to materialize. It is expected that this method result in lowering the overall cost of query answering on the temporal database.

		For the placement of a single snapshot for materialization, both mathematical and experimental approach showed that the most optimal timestamps is the median of previousely performed queries. This notion could be extended for placing multiple precomputed snapshots for materialization. That is, when optimal multiple segmentations of the queries were computed, the optimal place in each segmentation for placing the snapshot is the median of queries in that segment.

		Finding the optimal multiple segmentations of the previousely performed queries was performed using three different aproaches: recursive algorithm, dynamic programming and heuristic method. The experiments showed that the heuristic method does not guarantee the most optimal solution, but it is more favorable than the other two methods because of less computational time and satisfactory results.

		To extend the idea of snapshot materialization for the problem of Blockchain verification, it is recommended to create a digital signature of the precomputed snapshots, and place it to the end of each snapshot table. By doing so, the validation of the Blockchain, when a query is performed is done by:

		\begin{itemize}
			\item check the digital signature of the snapshot that the query is materializing.
			\item for the records that fall in between the query timestamp and snapshot timestamp, use the manual chain validation technique
		\end{itemize}


	\section{Conclusion}
		In this thesis, the development of a trusted relational temporal database was discussed. The temporal database contains the historical data of a database which makes them a suitable source of data provenance for the database system. In order for the temporal database to be trusted, the records in it must be immutable and verifiable. For this purpose, digital signature is a suitable tool to be utilized. Digital signatures provide verifiability for the stored records and also make undetectable forgeires on the temporal database difficult. Using digital signatures in the context of this research is achieved by creating a signature of the submitted transaction using the submitter's private key. Moreover, To reduce the risk of adverserial attacks, especially from the super-users of the system, the records that are digitally signed could be chained together by a record holding its previous record's digital signature. By doing so, any malicious or accidental data manipulation on the temporal relation, results in an inconsistent data.

		The temporal databases that we talked about can become massive as they store all the updates on the relations of a database. This makes querying for the snapshot of a relation on them expensive as there is a need to compute a large workload for each query. Snapshot materialization is a suitable way to reduce the cost of query answering on the temporal databases. For this purpose, multiple number of snapshots could be precomputed for materialization. Finding the optimal timestamps to place the snapshots could be started by storing the timstamp of performed queries on the temporal relation. By doing so, a useful information about the hotspots of the timeline of temporal table is provided. If one snapshot is to be placed, the median of queries is the most optimal place for it. For multiple snapshots, optimally creating segmentations of the queries and placing snapshots in the median of each segmentation, might an optimal solution.
		
		Creating optimal segmentation could be seen as an optimization problem and could be computed using recursive algorithm, dynamic programming and heuristic method. Experiments proved that in this project, the heuristic method is more favorable because it provides satisfactory results in a much affordable runtime. Multiple experiments also proved our claim that the optimal snapshot materialization can reduce the cost of query answering more significant than placing materialized snapshots randomly or in fixed intervals.

		At last, to guarantee that the snapshots created using materialized view is trustworthy, digital signature of the materialized snapshot could be created and added to their table. This is a beneficial method that reduces the cost of Blockchain verification by materializing the snapshots.

	\section{Future work and other remarks}
		Many different extensions and changes to the proposed system has been left for the future work due to the complicateness of the problems and the limitation of time. Future works contain different possible applications of the proposed system and methods which can make the current system perform better. For example, this thesis has mainly focused on the installation of the system in a computer system with central storage. This makes the application of the system limited and contradictory to the goal of building a relational database as open and as scalable as possible. Therefore, it would be interesting to consider the following applications and extensions of the proposed system.
		\subsection {Utilization in distributed storages on P2P networks}
			In recent years decentralized file systems have become a trend in business environments. Unlike centralized storages in which all data of a system stored on central servers and any data access should be authorized by these central servers, in decentralized file systems, computers can share information with each other directly without having an intermediary involved \cite{koonce2016thewild}. Current decentralized file systems such as {\it Interplanetrory File System} (IPFS), {\it Storj} and {\it Sia} are distributed storages on peer to peer networks that do not rely on any central service provider to monitor and permit file sharing between the nodes \cite{wang2018ablockchain}. 

			Many of decentralized file systems utilize Blockchain technology in order to make digital assets that are shared between the nodes secure and virtually immutable \cite{wang2018ablockchain}. For example, IPFS utilizes {\it Filecoin} which is basically a Blockchain that is built on top of IPFS and guarantees secure file replication and synchronization between the nodes \cite{vatsalya2018marinehull}.

			It would be interesting to employ the proposed trusted temporal relation to not only provide data provenance for the records in such decentralized file systems but also make queries on the relation as well as verifying the authenticity of the Blockchain computationally cheaper. However, in order to achieve the objective there are multiple problems that need to be solved.

			Finding the optimal snapshot materialization strategy in the distributed file system is the most flagrant problem. In this thesis we argued that the optimal snapshots could be placed in the center of hotspots on the timeline of the temporal database. We discussed that these hotspots could be identified by capturing the timestamp of past queries. However the usefulness of this strategy for a decentralized network needs to be investigated. It needs to be inquired if it is optimal to either place snapshots for every individual node on the decentralized network or it should be calculated once for all the nodes. The other possible question is where to store these snapshots on the network. All mentioned and many other problems create a new dimension to the optimal snapshot materialization problem.

		\subsection {Trust certification}
			In the world wide web environment, in order to provide secure connections between two parties in the network, {\it Secure Socket Layer} (SSL) certificates are used that encrypts all the interactions between a browser and a web server. Similar to that, based on our proposed system, a user interface with certification could be created that reflects the "{\it Trust}" in the data exploration. This cerificate guarantees that the stored data of a database and any results from queries on them are trustworthy. 
