\chapter{Conclusion}

	\section{Summary}
		In this work, implementation of a Blockchain based tool was proposed which works on top of the Relational Database Management System (RDBMS) and ensures the trustworthiness of the records that are stored in the relational database tables. This system follows a mechanism to store the historical records of the transactions on the relational tables in the temporal databases which can be used as a source of data provenance for the database system. To implement the system, two major problems was identified: 
		\begin{itemize}
			\item Lower the cost of queries performed on the temporal database.
			\item Add immutable proof of work to the stored reocords in the temporal table.
		\end{itemize}

		The problem of linear time and storage to perform queries on the temporal databases shown both theoretically and experimentally. In the presence of multiple and concurrent queries on the temporal table, performing such queries are inefficient and infeasable. To solve this problem, the materialization of multiple precomputed snapshots proposed. The proposal advices that keeping the timestamp of previous queries which was performed on the temporal table
		gives insight into the pattern of queries on the table and identifies hotspots. These hotspots are useful because there is a high chance that the subsequent queries to be performed in those hotspots. Optimal segmentation of the performed queries, clusters the hotspots and specifies a precomputed snapshot for each cluster to materialize. It is expected that this method result in lowering the overall cost of query answering on the temporal database.

		For the placement of a single snapshot for materialization, both mathematical and experimental approach showed that the most optimal timestamps is the median of previousely performed queries. This notion could be extended for placing multiple precomputed snapshots for materialization, that is when optimal multiple segmentations of the queries were computed, the optimal place in each segmentation for placing the snapshot is the median of queries in that segment.

		Finding the optimal multiple segmentations of the previousely performed queries was performed using three different aproaches: recursive algorithm, dynamic programming and heuristic method. The experiments showed that although the heuristic method does not guarantee the most optimal solution, but because of less computational time in comparison with other two methods and returning satisfactory results, it is more favorable to be utilized in this project.

		Using Blockchain to add immutable proof of work for the records of temporal tables, require these tables to have a few mandatory attributes: the transaction submitters information, the digital signature of the record signed by transaction submitter's cryptoraphic keys and the previous record's digital signature. Having the digital signature of the previous record chains the submitted records together and makes any malicious or accidental record modification evident.

		The validation of the chain of records is done by
		\begin{itemize}
			\item making sure that the current record's \it{'previous signature'} matches the previous record's \it{'current signature'} attribute for all the records in the table.
			\item validate the signature of every single record by using the transaction submitters cryptographic keys.
		\end{itemize}
		
		This procedures are expensive and contradicts the idea of snapshot materialization that we discussed earlier. To solve the problem, it is recommended to create a digital signature of the precomputed snapshots, then the validation of the chain of records is done by:
		\begin{itemize}
			\item check the digital signature of the snapshot that the query is materializing.
			\item for the records that fall in between the query timestamp and snapshot timestamp, use the manual chain validation technique
		\end{itemize}

		Digitally signing the snapshots require the system to check the authenticity of the records that are stored in it beforehand. This procedure requires the system to manually check the validation of the chain of records. This is an expensive procedure, specially for the snapshots that are placed at the end of the timeline. To reduce the cost, it was proposed that the manual chain validation to be performed for the first snapshot and the subsequent snapshots materialize their previous snapshot for chain validation.

	\section{Conclusion}
		In this thesis, in order to establish a trusted temporal relational database, the use of historical data as a source of data provenance is suggested. Data provenance sources can hold useful evidences about the legitimacy of a record throughout its lifecycle in a database system. Temporal databases are a good candidate to store the historical data on becuase it gives us the flexibility to implement our security objectives for the records. Since the temporal databases are not immune from malicious attacks, it is required that the data stored in them to be immutible, and any attempts for a forgery on their records to be computationally expensive. Digital signature not only guarantees the immutability of the records in a database, but also is a verifiable evidence that proves the authenticity of the transactions. This makes digital signatures as a perfect tool to guarantee the immutability of the records in the source of data provenance. To reduce the risk of adverserial attacks, the records that are digitally signed could be chained together by a record holding its previous record's digital signature. This makes any malicious or accidental modification of the records evident.
		Append-only temporal databases can become massive as they store all the updates on the relations of a database. This makes querying on the temporal database expensive as there is a need to compute a large workload for each query. Snapshot materialization is a suitable way to reduce the cost of query answering on the temporal databases. If the right number of snapshots chosen, and then placed in optimal timestamps for materialization, not only the cost of answering to the queries can be reduced, but also the verification of the authenticity of chains of digitla signatures can be done in a much faster way.

	\section{Future work and other remarks}
		Many different extensions and changes to the proposed system has been left for the future work due to the complicateness of the problems and the limitation of time. Future works contain different possible applications of the proposed system and methods which can make the current system perform better. For example, this thesis has mainly focused on the installation of the system in a computer system with central storage. This makes the application of the system limited and contradictory to the goal of building a relational database as open and as scalable as possible. Therefore, it would be interesting to consider the following applications and extensions of the proposed system.
		\subsection {Utilization in distributed storages on P2P networks}
			In recent years decentralized file systems have become a trend in business environments. Unlike centralized storages in which all data of a system stored on central servers and any data access should be authorized by these central servers, in decentralized file systems, computers can share information with each other directly without having an intermediary involved \cite{koonce2016thewild}. Current decentralized file systems such as {\it Interplanetrory File System} (IPFS), {\it Storj} and {\it Sia} are distributed storages on peer to peer networks that do not rely on any central service provider to monitor and permit file sharing between the nodes \cite{wang2018ablockchain}. 

			Many of decentralized file systems utilize Blockchain technology in order to make digital assets that are shared between the nodes secure and virtually immutable \cite{wang2018ablockchain}. For example, IPFS utilizes {\it Filecoin} which is basically a Blockchain that is built on top of IPFS and guarantees secure file replication and synchronization between the nodes \cite{vatsalya2018marinehull}.

			It would be interesting to employ the proposed trusted temporal relation to not only provide data provenance for the records in such decentralized file systems but also make queries on the relation as well as verifying the authenticity of the Blockchain computationally cheaper. However, in order to achieve the objective there are multiple problems that need to be solved.

			Finding the optimal snapshot materialization strategy in the distributed file system is the most flagrant problem. In this thesis we argued that the optimal snapshots could be placed in the center of hotspots on the timeline of the temporal database. We discussed that these hotspots could be identified by capturing the timestamp of past queries. However the usefulness of this strategy for a decentralized network needs to be investigated. It needs to be inquired if it is optimal to either place snapshots for every individual node on the decentralized network or it should be calculated once for all the nodes. The other possible question is where to store these snapshots on the network. All mentioned and many other problems create a new dimension to the optimal snapshot materialization problem.

		\subsection {Trust certification}
			In the world wide web environment, in order to provide secure connections between two parties in the network, {\it Secure Socket Layer} (SSL) certificates are used that encrypts all the interactions between a browser and a web server. Similar to that, based on our proposed system, a user interface with certification could be created that reflects the "{\it Trust}" in the data exploration. This cerificate guarantees that the stored data of a database and any results from queries on them are trustworthy. 
