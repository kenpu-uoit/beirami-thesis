This thesis deals with the development of a Blockchain-based framework which is built on top of a Relational Database Management System (RDBMS) to discover and report authenticity of the stored records as well as malicious data manipulation on the database system. To do this, we proposed a mechanism to document and analyze the historical records of a database. Historical records are gathered by auditing the executive activities on the tables of the database system and storing these activities in the tamper-evident temporal log tables. The temporal log tables utilize Blockchain technology for two purposes. first, to analyze and testify the authenticity of the records stored in the normal tables and second, to make the log tables tamper-evident. This thesis argues that the Blockchain technology makes it extremely difficult for the privileged users or outsiders to conceal their activities on a database.  
Although logging the transactions is very useful for forensic purposes, the fact that they require linear time to query the historical data of records, is seen as a challenging problem. To address this issue, optimally creation of multiple snapshots for the materialization is suggested to lower the time complexity of querying for the historical records on the log tables.
The proposed system supports all RDBMS systems and to achieve its objectives, the system solely rely on verifiable evidences and not on putting trust on the users of the system.
